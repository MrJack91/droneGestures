%\chapter{Beschreibung der Aufgabe}
\chapter{Einführung}

\section{Allgemein}

\subsection{Zielpublikum}
Obwohl im Glossar einige Fachbegriffe erklärt werden, wird für das Verstehen dieser Arbeit technisches Verständnis vorausgesetzt.

Eine konsequente Übersetzung englischer Begriffe würde keinen Beitrag bezüglich Verständlichkeit leisten und wurde daher weggelassen.

\subsection{Ressourcen}
Weitere Dateien, wie der Source Code der Steuerung und der \LaTeX-Quelltext dieses Dokuments, sind auf GitHub verfügbar.\footcite{github_droneGestures_2015-05-01}


\section{Aufgabenstellung}
\label{sec:workdef}
Folgende Aufgabenstellung wurde direkt aus dem \gls{ebsLabel} entnommen.

\subsection{Ausgangslage}
Heutzutage nehmen die Möglichkeiten der Technik sehr rasch zu. Es kommen immer mehr erschwingliche, neue Technologien auf den Markt. Besonders für Entwickler bietet dies vielseitige und interessante Möglichkeiten.

Die Firma Bitcraze stellt programmierbare Drohnen zur Verfügung. Ebenso kann von der Firma Leap Motion ein Handerkennungs-Sensor bezogen werden.

\subsection{Ziel der Arbeit}
Eine programmierbare Drohne (Crazyflie Nano Quadcopter von Bitcraze) soll mit einer Gestensteuerung per Hand (von Leap Motion) ergänzt werden.
Die Drohne soll intuitiv und einfach mit Möglichkeiten der Hand-/Gestenerkennung gesteuert werden können.

\subsection{Begründung}
Die Erkennung von Gesten wird laufend verbessert, trotzdem findet man Gestenanwendungen vor allem bei Computer-Spielen.
Gerade weil Gesten aus Benutzersicht sehr intuitiv verwendet werden können, lohnt es sich weitere Anwendungen mit Gesten umzusetzen.

Dazu bieten sich Steueraufgaben an.
Der Umfang einer Steuerung ist relativ einfach zu definieren, da meist bereits alternative Steuermöglichkeiten umgesetzt worden sind und daher der grundlegende Funktionsumfang mehrheitlich übernommen werden kann.

Die Idee einer Umsetzung für eine Drohnensteuerung entstand dadurch, dass das Fliegen die Menschen schon seit langem begeistert und Drohnen in letzter Zeit günstig bezogen werden können.

Die Verbindung zweier existierenden Systeme und die Umsetzung der Logik einer kompletten Flugsteuerung, entspricht einer sehr interessanten Aufgabe mit viel Begeisterungspotenzial.


\subsection{Aufgabenstellung}
\subsubsection{Recherche}
\begin{itemize}
	\item Geschichte der Drohnen
	\item Flugeigenschaften der Drohnen
	\item bereits bestehende Gestensteuerung für Drohnen
	\begin{itemize}
		\item Umsetzungsart
		\item Steuermöglichkeiten
		\item Umsetzungen mit gleicher Hardware (Leap Motion, BitCraze Crazyflie 2.0)
	\end{itemize}
\end{itemize}


\subsubsection{Ist-Analyse}
\begin{itemize}
	\item Gesten Sensor
	\begin{itemize}
		\item Einsatz
		\item Technische Details
		\item API
	\end{itemize}
	
	\item Drohne analysieren
	\begin{itemize}
		\item Einsatz
		\item Technische Details
		\item API
	\end{itemize}
\end{itemize}


\subsubsection{Soll-Analyse}
\begin{itemize}
	\item Definition einer gestensteuerbare Drohne (Yaw, Pitch, Roll, Thrust)
\end{itemize}


\subsubsection{Konzept}
\begin{itemize}
	\item detailierter Gesten-Steuerbeschrieb
	\item Mögliche Probleme behandeln (mind. schriftlich)
	\begin{itemize}
		\item Gestenerkennung
		\item Kommunkationsprobleme
	\end{itemize}
	\item (für die erwähnten Punkte sollen bereits Testszenarien erfasst werden)
\end{itemize}


\subsubsection{Proof-of-Concept}
\begin{itemize}
	\item Umsetzung der Gestensteuerung
\end{itemize}


\subsubsection{Testing}
\begin{itemize}
	\item Erstellung und Durchführung des Testplans (Testszenarien aus Konzept durchführen)
	\item Steuerung stand-alone testen (nur Konsolen-Output)
\end{itemize}

\subsection{Erwartete Resultate}
\subsubsection{Recherche}
\begin{itemize}
	\item Erwähnte Themen in der Aufgabenstellung müssen schriftlich abgedeckt werden
\end{itemize}

\subsubsection{Ist-Analyse}
\begin{itemize}
	\item Gesten-Sensor soll beschrieben sein
	\item Drohne (inkl. Sensoren) soll beschrieben sein
\end{itemize}

\subsubsection{Soll-Analyse}
\begin{itemize}
	\item Schriftliche Definition einer Umsetzung für eine gestensteuerbare Drohne
\end{itemize}

\subsubsection{Konzept}
\begin{itemize}
	\item Schriftlicher, detaillierter Gestensteuerbeschrieb
	\item Schriftliche Auflistung möglicher Probleme
\end{itemize}

\subsubsection{Proof-of-Concept}
\begin{itemize}
	\item Gestensteuerbare Drohne
\end{itemize}

\subsubsection{Testing}
\begin{itemize}
	\item Auswertung des Testplanes
	\item Gesten-Steuersimulation (Konsolen-Output)
\end{itemize}


\subsection{Eingrenzungen und Abgrenzungen}
Die Arbeit umfasst den Prozess vom Konzept bis zur erfolgreichen Implementierung der Gestensteuerung einer Drohne.
Dabei steht der Fokus auf der Umsetzung der Steuerung.

Es werden externe Systeme (die Drohne und der Gestensensor) verwendet.
Die externen Systeme werden beschrieben und die verwendeten Funktionen genau erläutert.
Auf technische Details wie Funkverbindung, interne Drohnensteuerung, effektiver Gestenerkennungsablauf etc. wird bewusst nur oberflächlich eingegangen, da dies der Zeitrahmen der Arbeit überschreiten würde.
Da die externen Systeme eigenständige Produkte sind, können Informationen zu jenen technischen Details beim jeweiligen Hersteller bezogen werden.
Mögliche grundsätzliche Fehlerquellen dieser externen Systeme werden ebenfalls vom Hersteller abgedeckt.
Das grundlegende Funktionieren kann somit als gegeben betrachtet werden.

Selbstverständlich werden Fehlerquellen, die zusätzlich durch die Gestensteuerung entstehen, behandelt.
