\chapter{Beschreibung der Aufgabe}

\section{Aufgabenstellung}
% gemäss EBS
Folgende Aufgabenstellung wurde direkt aus dem \gls{ebsLabel} entnommen.

\subsection{Ausgangslage}
In letzter Zeit nehmen die Möglichkeiten der Technik sehr rasch zu. Es kommen immer mehr erschwingliche, neue Technologien auf den Markt. Besonders für Entwickler, bietet dies vielseitige und interessante Möglichkeiten.

So stellt die Firma Bitcraze programmierbare Drohnen zur Verfügung. Ebenso kann von der Firma Leap Motion ein Handerkennungs-Sensor bezogen werden.

\subsection{Ziel der Arbeit}
Eine programmierbare Drohne (Crazyflie Nano Quadcopter von Bitcraze) soll mit einer Steuerung per Hand (Gestensteuerung von Leap Motion) ergänzt werden.
Die Drohne soll intuitiv und einfach mit Möglichkeiten der Hand-/Gestenerkennung gesteuert werden können.

\subsection{Aufgabenstellung}
\subsubsection{Recherche}
\begin{itemize}
	\item Geschichte der Drohnen
	\item Flugeigenschaften der Drohnen
	\item bereits bestehende Gestensteuerung für Drohnen
	\begin{itemize}
		\item Umsetzungsart
		\item Steuermöglichkeiten
		\item Umsetzungen mit gleicher Hardware (Leap Motion, BitCraze Crazyflie 2.0)
	\end{itemize}
\end{itemize}


\subsubsection{Ist-Analyse}
\begin{itemize}
	\item Gesten Sensor
	\begin{itemize}
		\item Einsatz
		\item Technische Details
		\item API
	\end{itemize}
	
	\item Drohne analysieren
	\begin{itemize}
		\item Drohnen Sensoren charakterisieren (Analyse, Tests)
			\begin{itemize}
				\item Accelerometer, Gyrometer, Magnetometer, Pressure
			\end{itemize}
	\end{itemize}
	
	\begin{itemize}
		\item Bestehende Steuerungen auf Flugstabilität analysieren
		\begin{itemize}
			\item Inwiefern werden die Sensoren zur Flugstabilität genutzt?
		\end{itemize}
	\end{itemize}
\end{itemize}


\subsubsection{Soll-Analyse}
\begin{itemize}
	\item Definition einer gestensteuerbare Drohne (Yaw, Pitch, Roll, Thrust)
\end{itemize}


\subsubsection{Konzept}
\begin{itemize}
	\item detailierter Gesten-Steuerbeschrieb
	\item Mögliche Probleme behandeln (mind. schriftlich)
	\begin{itemize}
		\item Gestenerkennung
		\item Kommunkationsprobleme
	\end{itemize}
	\item (für die erwähnten Punkte sollen direkt schon Testszenarien erfasst werden)
\end{itemize}


\subsubsection{Proof-of-Concept}
\begin{itemize}
	\item Umsetzung der Gestensteuerung
\end{itemize}


\subsubsection{Testing}
\begin{itemize}
	\item Erstellung und Durchführung des Testplans (Testszenarien aus Konzept durchführen)
	\item Steuerung stand-alone testen (nur Konsolen-Output)
\end{itemize}

\subsection{Erwartete Resultate}
\subsubsection{Recherche}
\begin{itemize}
	\item Erwähnte Themen in der Aufgabenstellung müssen schriftlich abgedeckt werden.
\end{itemize}

\subsubsection{Ist-Analyse}
\begin{itemize}
	\item Gesten-Sensor soll beschrieben sein
	\item Drohne (inkl. Sensoren) soll beschrieben sein
\end{itemize}

\subsubsection{Soll-Analyse}
\begin{itemize}
	\item Schriftliche Definition einer Umsetzung für eine gestensteuerbare Drohne
\end{itemize}

\subsubsection{Konzept}
\begin{itemize}
	\item Schriftlicher detailierter Gestensteuerbeschrieb
	\item Schriftliche Auflistung möglicher Probleme
\end{itemize}

\subsubsection{Proof-of-Concept}
\begin{itemize}
	\item Gestensteuerbare Drohne
\end{itemize}

\subsubsection{Testing}
\begin{itemize}
	\item Auswertung des Testplanes
	\item Gesten-Steuersimulation (Konsolen-Output)
\end{itemize}


\subsection{Eingrenzung}
%todo 

\subsection{Abgrenzung}
%todo 

