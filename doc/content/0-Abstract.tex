% !TeX encoding=utf8
% !TeX spellcheck = de_CH_frami

\subsection*{Abstract}
In diesem Dokument wird die Umsetzung einer Gestensteuerung für eine Drohne beschrieben. Als Drohne wird der Crazyflie 2.0 Nano Quadrocopter von Bitcraze verwendet. Für die Gestenerkennung wird auf den Leap Motion Sensor zurückgegriffen.

Das Hauptziel der Steuerung ist die intuitive Bedienung, die aus einfachen Handgesten abgeleitet werden kann. Die Umsetzung soll die Vorteile, so wie auch möglichen Probleme einer Gestensteuerung aufzeigen. Zusätzlich zur Steuerung, wird auf die Problematik der Initialisierung der Gestensteuerung eingegangen. Wie kann sichergestellt werden, dass unabsichtlich ausgelöste Gesten keine Auswirkungen auf die Drohne haben? Nebst dem Konzept wird eine Umsetzung anhand der Crazyflie und dem Leap Motion Sensor in Python implementiert.

Für die Erarbeitung einer Gestensteuerung ist eine klare Gliederung unabdingbar. So ist zu Beginn eine gründliche Recherche der Beschaffenheit von Quadrocoptern notwendig. Nur so kann anschliessend eine Ist- und eine Soll-Analyse erstellt werden. Aufgrund dieser Analyse lässt sich das Konzept der Gestensteuerung definieren, sowie die auf erster Stufe erkannten Probleme bereits zu eruieren.

Auf der Basis der konkreten Implementierung, welche durch Tests verifiziert wurde, können nun Rückschlüsse auf das Konzept gewonnen werden, wodurch dieses bei Bedarf präzisiert werden kann.

Nur so kann unter Berücksichtigung aller Vor- und Nachteile einer Gestensteuerung eine Schlussfolgerung erarbeitet und ein Fazit gezogen werden.

% Eine programmierbare Drohne (Crazyflie Nano Quadrocopter von Bitcraze) soll mit einer Steuerung per Hand (Gestensteuerung von Leap Motion) ergänzt werden. Die Drohne soll intuitiv und einfach mit Möglichkeiten der Hand-/Gestenerkennung gesteuert werden können.
\begin{flushright}
	\textit{Michael Hadorn}	
\end{flushright}

\vfill

%
\mbox{}\\[0.5\baselineskip]\noindent
\textbf{Schlagwörter:} 
Leap Motion, Crazyflie 2.0, Gestensteuerung, Quadrocopter, Drohne

\newpage
\subsection*{Versionsübersicht}
\begin{center}
	\centering
	\small\renewcommand{\arraystretch}{1.4}
	\rowcolors{1}{tablerowcolor}{tablebodycolor}
	\begin{tabularx}{1.0\textwidth}{ R{0.1\linewidth} | R{0.16\linewidth} | X  }%
		\hline
		\textbf{Version*} & \textbf{Datum} & \textbf{Kommentar}\\
		\hline
		- & 09.02.15 & Dokument Erstellung \\
		v0.0.1 & 20.03.15 & 1. Abgabe: Steuerungsbeschrieb\\
		v0.0.2 & 30.03.15 & 2. Abgabe: Recherche \\
		v0.0.3 & 01.05.15 & 3. Abgabe: Ist-Analyse \\
		v0.0.4 & 31.05.15 & 4. Abgabe: Soll-Analyse \\
		v0.0.5 & 03.08.15 & 5. Abgabe: Proof of Concept, Kosten \\
		v0.1.0 & 04.09.15 & Inhalt vollständig \\
		v1.0.0 & 06.09.15 & Finale Version, Korrektur \\
		\hline
	\end{tabularx}
\end{center}
\vspace{-1.0\baselineskip}
{\footnotesize * Eine genauere Übersicht der Änderungen kann auf GitHub eingesehen werden.\footcite{github_droneGestures_2015-05-01}}

\vfill

\subsection*{Danksagung}
Mein besonderer Dank gilt Sina Masquiren, Jonathan Hadorn und Karin Hadorn für die vollständige und detaillierte Korrekturlesung und das geteilte Interesse an den durch diese Arbeit gewonnenen Erkenntnissen von mir.\\
Ebenfalls bedanke ich mich für die Betreuung dieser Arbeit bei Prof. H. Doran.
