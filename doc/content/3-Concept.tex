\chapter{Konzept}

\section{Detaillierter Gesten-Steuerbeschrieb}

Gestensteuerungen sind für "`gewöhnliche"' Steueraufgaben nach wie vor sehr wenig verbreitet, obwohl es theoretisch nichts intuitiveres als Gesten gibt.
Dies kann unter Anderem an fehlenden Anwendungsfällen oder an falsch angesetzten Umsetzungen liegen.

Ziel dieser Steuerung ist es, die Drohne sehr einfach steuern zu können.
Das auswendig Lernen der möglichen Gesten soll durch intuitive Schlussfolgerungen des Anwendungswunsches entfallen.
Der Grundgedanke der Steuerung soll die Hand sein, welche die Drohne verkörpert.

\subsection{LeapMotion}
Es folgen grobe Informationen zum LeapMotion.

Der Sensor erkennt problemlos Gesten innerhalb eines 150\textdegree-Winkels und zwischen einer Höhe von 25 mm bis 600 mm.
Sprich auf ca. 30 cm Höhe funktioniert die Erkennung bestens und wir haben genügend vertikale Spatzung um Höhenunterschiede festzustellen.
Im Folgenden wird auf die grundlegenden Flugmanöver eingegangen.


\subsection{Flugmanöver}

\subsubsection{Init}

Da die Hand die Drohne repräsentieren soll, muss eine initiale Position eingenommen werden, aus dieser die Drohne anschliessend (relativ gemessen) gesteuert werden kann.

Dieser Prozess vom Finden der initialen Position wird folgend als \textit{Init-Prozess} bezeichnet und sieht wie folgt aus:

\begin{itemize}
	\item Die Drohne befindet sich im \textit{Init-Zustand}.
	\item Die Hand wird auf idealer Höhe platziert (ca. 10-20 cm über dem Sensor).
	\item Die Position gilt als eingenommen, sobald nach einer Faust wiederum die offene Hand erkannt wird. (Faust-Gesten in nicht idealen Höhen werden ignoriert.)
	\item Die Drohne befindet sich nun im \textit{flugbereiten-Zustand} (die Rotoren drehen nicht).
	\item Die Hand bleibt geöffnet (erneute Faust-Geste wechselt vom \textit{flugbereiten-Zustand} zurück in den \textit{Init-Zustand}).
\end{itemize}

\subsubsection{Start / Thrust}
Um mit der Drohne abzuheben wird die offene Hand nach oben bewegt. Dabei gilt, desto höher die Hand zur relativen Position gehoben wird, umso mehr drehen die Rotoren und umso schneller bewegt die Drohne sich nach oben (dies entspricht dem \textit{Thrust}).

Umgekehrtes gilt auch: wird die Hand gesenkt, senkt sich auch die Drehzahl der Rotoren, sprich die Drohne verliert an Steigung.
Der \textit{Thrust} kann während des ganzen Fluges mit diesen Bewegungen gesteuert werden.
Die Drohne befindet sich, sobald abgehoben, im \textit{Flug-Zustand} (mit drehenden Rotoren und in der Luft).

\subsubsection{Flug}
Während dem Flug kann nebst dem \textit{Thrust}, die Drehung (\textit{Yaw}) und die Neigung (\textit{Pitch/Roll}) gesteuert werden.
Auch hier gilt jeweils, dass die Drohne sich nach der Hand ausrichtet.
Die Drohne ist solange im \textit{Flug-Zustand} bis sie wieder auf festem Untergrund ist, dann ist sie wieder im \textit{flugbereiten-Zustand}.
Bezüglich der Steuermöglichkeiten ändert sich weder im \textit{Flug-} noch im \textit{flugbereiten-Zustand} etwas.
Sprich es ist kein Zustandswechsel nötig, um die Drohne zu landen.

\textbf{Bemerkung zur Steuerung:} Grundsätzlich wird der Level (Power) einer Steueranweisung, gemäss der Abweichung der natürlichen, rechtwinkligen, vertikalen Achsen gesetzt.
Wobei ein ca. 60\textdegree-Abweichung dem Maximum entspricht.

\subsubsection{Yaw}
Analog zum Thrust bedeutet eine flache Drehung der Hand (Drehung an der vertikalen Rotations-Achsen) eine entsprechende Richtungsänderung der Drohne.

\subsubsection{Pitch (Neigung) / Roll (Rolle)}
Beim Pitch und Roll dreht sich die Drohne um eine horizontale Rotations-Achse. Beim Pitch hebt oder neigt die Drohne ihre Front. Bei Roll sinkt oder erhebt sich ihre eine Seite.

\subsubsection{Finish}
Befindet sich die Drohne im \textit{flugbereiten-Zustand}, kann via eine Faust-Geste, die Steuerung abgebrochen werden.
Die Drohne reagiert bis zum nächsten \textit{Init-Prozess} nicht.


\subsection{Zustands Übersicht}
Die Drohne, insofern eingeschaltet, kann drei Zustände einnehmen:

\subsubsection{Init-Zustand}
Die Drohne kann nicht geflogen werden, sondern die Hand muss zuerst die initiale Position einnehmen.

\subsubsection{Flugbereiten-Zustand}
Die Drohne fliegt nicht, hat aber eine gültige Hand an der Steuerung. Dies kann nach und vor einem Flug sein.

\subsubsection{Flug-Zustand}
Die Drohne befindet sich in der Luft.


%%% 

%todo
\section{Problembehandlung}
Während den verschiedenen Zustände können verschiedene Fehler auftreten.

\subsubsection{Init-Zustand}
Problem:        Faust wird nicht erkannt oder mehrere Hände werden erkannt
Gefahr:        gering
Folge:                Die Drohne kann nicht abheben. Der Init Zustand wird belassen.

\subsubsection{Flugbereiten-Zustand}
Problem:        Die Hand wird nicht mehr erkannt
Gefahr:        gering
Folge:                Die Drone setzt sich zurück in den Init Zustand. (Die Drohne befindet sich noch auf dem Boden.)

\subsubsection{Flug-Zustand}
Problem:        Die Hand wird nicht mehr erkannt oder mehrere Hände werden erkannt
Einstufung:        sehr gross
Verhalten:        Der Thrust wird sofort auf mässiges Level getrimmt. Falls länger als t Sekunden keine Hand erkannt wird, werden die Rotoren ausgeschaltet. Die Drohne wechselt in den Init Zustand.
Folge:                Die Drohne befindet sich ungesteuert in der Luft oder wird notgelandet.

Problem:        grosse ruckartiger Gestenwechsel (Angst-Reaktion oder ähnliches)
Gefahr:        erheblich
Verhalten:        Die Steuerung reguliert, resp. vermeidet sprunghafte Steuerbefehle.
Folge:                Kunst Flüge mit bewussten schnellen Anweisungen sind nicht möglich.

\section{Tests}


%%% 
