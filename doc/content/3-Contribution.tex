
% Eigener Beitrag: Beschreibung, Begründung, Aufzeigung, Methode, Fazit

\chapter{Analyse}
\label{sec:analyse}

% ...
\section{Allgemein}
% Amazon Lösung: https://aws.amazon.com/marketplace/pp/B00FGB3528
	% https://aws.amazon.com/marketplace/pp/B00GDZINWI
% Free VPS https://manage.haphost.com
% VMware: http://www.vmware.com/products/workstation
% Microsoft: http://www.microsoft.com/en-us/server-cloud/products/virtual-desktop-infrastructure/default.aspx
	% http://www.microsoft.com/en-us/windows/enterprise/products-and-technologies/virtualization/default.aspx
	% Microsoft Azure
	% Microsoft HyperV: http://www.microsoft.com/en-us/server-cloud/solutions/virtualization.aspx

Folgend werden verschiedene Möglichkeiten von IT-Infrastrukturen vorgestellt.
Besonders auf zwei cloudbasierte Lösungen wird genauer eingegangen. Dazu wird je ein Vertreter aus der Marktwirtschaft analysiert.

\section{Typen}
Grundsätzlich kann ein Arbeitsplatz via virtuellen Computern in der \Gls{glos:cloudLabel} oder klassisch, \gls{glos:onPremiseLabel} umgesetzt werden.

\section{Klassischer Ansatz, "`on-premise"'}
Beim klassischen Ansatz, wird jeder Client lokal auf eigener Hardware aufgesetzt. Wird öfters eine genau gleiche Umgebung eingerichtet, so kann mit einem Image eine vollständige Einrichtung eines Computers vorgenommen werden.

Eine IT-Infrastruktur, die auf \gls{glos:onPremiseLabel} Geräten basiert, kommt ohne \Gls{glos:cloudLabel} aus. Es ergibt sich eine dezentrale Sammlung von Daten (jeweils auf dem Gerät des jeweiligen Benutzers). Selbstverständlich können Daten auf Servern gesammelt und auch wieder automatisch verteilt werden, trotzdem müssen die Daten schlussendlich über die Hardware des Clients verarbeitet werden.

Die Ressourcen pro Arbeitsplatz sind auf deren Hardware beschränkt.
Eine Änderung der Hardware erfolgt durch eine manuell, mechanische Anpassungen des Gerätes.
Daraus folgt, dass umfassende Hardware-Anpassungen an mehreren Geräten sehr aufwändig und somit auch kostspielig sind.

\subsection{Cloud-Lösung}
Um die Arbeitsplätze einer Firma in die \Gls{glos:cloudLabel} zu heben, muss entschieden werden, ob man eine Lösung im eigenen Rechenzentrum oder bei einer externen Firma warten will.
Die Lösung im eigenen Rechenzentrum heisst im Fachjargon \Gls{vdiLabel} und beim Hosting bei einer externen Firma spricht man von \Gls{daasLabel}.

Beide Vorgehensweisen beinhalten Vorteile und Tücken. Allgemeine Vorteile, welche beide Arten mit sich bringen, sind bereits im \cref{sec:Vorteile} beschrieben.

Wird das Hosting ausgelagert, hat man eine monatliche oder jährliche Abrechnung. Dadurch werden die Kosten kalkulierbar. Auch fallen grosse initiale Anschaffungskosten weg.

Bei Firmen mit gesetzlichen Auflagen wie Versicherungen, Banken, etc. kann es dabei Probleme mit dem Datenschutz geben.
Dabei spielt die effektive Lagerung der Daten (Lokalität) und die Sicherheit (konsequente Verschlüsselung) deren Übertragung, sowie deren Speicherung eine zentrale Rolle.
Deswegen bietet sich ein eigenes Hosting an. Die Anschaffungskosten sind massiv höher und die Betriebskosten können volatil sein, man hat jedoch die volle Kontrolle über die Lokalität und Sicherheit.

Im Folgenden wird auf Amazon WorkSpaces, als Vertreter der ausgelagerten Lösung, sowie VMware Virtual Desktop Manager, für eine eigenständige Lösung eingegangen.

\newpage
\section{Amazon WorkSpaces}
% mh: http://aws.amazon.com/de/workspaces
% http://en.wikipedia.org/wiki/Amazon.com
% http://aws.typepad.com/aws/2013/11/tco-comparison-amazon-workspaces-and-traditional-virtual-desktop-infrastructure-vdi.html

Amazon startete als einfache Bücherverkaufsplattform, fügte jedoch bald weitere Artikel wie CDs, MP3s, Software etc. hinzu, wie auch eigene Produkte, wie Tablets und E-Books.
Zudem wird eine breite Plattform verschiedener \Gls{glos:cloudLabel} Computing Services, \Gls{awsLabel} genannt, angeboten.\footcite{Amazon.com_-_Wikipedia_the_free_encyclopedia_2014-11-15}

Die Lösung für Workstations in der \Gls{glos:cloudLabel} heisst \textit{Amazon WorkSpaces}\footcite{AWS_Amazon_WorkSpaces_2014-11-03}.
Mit Amazon WorkSpaces wird eine vollständige \Gls{daasLabel}-Lösung angeboten.

Leider können WorkSpaces nicht gratis getestet werden. Aus diesem Grund gibt es hier keine Tests und Erfahrungsberichte.

\subsection{Angebot}
Eine Workspace kann ab 35 USD pro Monat genutzt werden.
Auf diese kann per iPad, Android, Kindle Fire, PC und Mac zugegriffen werden.\\
Zur Auswahl stehen zwei Typen: \textit{Standard} und \textit{Leistung}.

\begin{table}[H]
	\centering
	\small\renewcommand{\arraystretch}{1.4}
	\rowcolors{1}{tablerowcolor}{tablebodycolor}
	%
	\captionabove[Amazon WorkSpace-Angebote]{Amazon WorkSpaces Angebots-Übersicht}
	%
	\begin{tabularx}{0.9\textwidth}{X | Y | Y | Y | Y }
		\hline
		\rowcolor{tableheadcolor}
		\textbf{WorkSpaces-Paket} & \textbf{vCPU}\footcite{Virtual_CPUs_with_Amazon_Web_Services_2014-11-15} & \textbf{RAM} & \textbf{Benutzerspeicher} & \textbf{Gebühren}\footnote{Preise entsprechen dem günstigsten Angebot (Hosting in der USA), je nach Standort können zusätzlich noch bis zu 18 USD anfallen.}\\
		\rowcolor{tableheadcolor}
		 & \# & GiB & GB & USD/mtl.\\
		\hline
			\textbf{Standard} & 1 & 3.75 & 50 & 35\\
			\textbf{Leistung} & 2 & 7.5 & 100 & 60\\
		\hline
	\end{tabularx}
\end{table}

Folgende Standard Software ist vorinstalliert:
\begin{itemize}
	\item Adobe Reader
	\item Internet Explorer 9
	\item Firefox
	\item 7-Zip
	\item Adobe Flash
\end{itemize}

Für zusätzliche 15 USD/Monat kriegt man weitere Software (inkl. Lizenzen) im "`Plus"'-Paket:
\begin{itemize}
	\item Microsoft Office Professional 2010
	\item Trend Micro Worry-Free Business Security Services
	\item WinZip
\end{itemize}

Als Standord der Hostingcentern kann zwischen folgenden fünf \Gls{awsLabel}-Regionen ausgewählt werden. Die Preise varieren je nach Standort.
\begin{itemize}
	\item USA Ost (Nord-Virginia)
	\item USA West (Oregon)
	\item EU (Irland)
	\item Asien-Pazifik (Sydney)
	\item Asien-Pazifik (Tokio)
\end{itemize}


\subsection{Aufwand}
Der Initial- und Wartungsaufwand (zeitlicher Aufwand, wie auch das Vorhandensein des technisches Verständnisses) sind deutlich geringer, als wenn auf andere Weise eine gleichwertige Lösung angeboten werden sollte.\footnote{Auf Vor- und Nachteile bezüglich \textit{Bring your own Device} wird in diesem Dokument nicht eingegangen.}
Es muss lediglich eine Internetverbindung angeboten werden.
Das Management der Workspace ist deutlich einfacher, da keine Anforderungen an Server Hardware, internes Netzwerk, Backup Lösung und Server Umgebung gestellt werden.


\subsection{Kosten}
Die Kosten einer Workspace variieren je nach \Gls{awsLabel}-Region.\footcite{AWS_Amazon_WorkSpaces_Preise_2014-11-15}

\begin{table}[H]
	\centering
	\small\renewcommand{\arraystretch}{1.4}
	\rowcolors{1}{tablerowcolor}{tablebodycolor}
	%
	\captionabove[Amazon WorkSpaces Preise]{Amazon WorkSpaces Preis-Übersicht nach Region}
	%
	\begin{tabularx}{0.9\textwidth}{X | Y | Y | Y | Y | Y}
		\hline
		\rowcolor{tableheadcolor}
		\textbf{Paket} & \textbf{Nord-Virginia} & \textbf{Oregon} & \textbf{Irland} & \textbf{Sydney} & \textbf{Tokio}\\
		\hline
		\textbf{Standard} & 35 & 35 & \textbf{37} & 45 & 47\\
		\textbf{Leistung} & 60 & 60 & \textbf{64} & 75 & 78\\
		\hline
		\multicolumn{6}{r}{
			Bemerkungen: alle Preise in USD, Stand: 17. 11. 2014
		}
	\end{tabularx}
\end{table}

Wie bereits erwähnt, kann das Software Paket "`Plus"' für zusätzliche 15 USD pro Monat beansprucht werden.

Um eine Kosteneinsparung gegenüber einer internen \Gls{vdiLabel}-Lösung auszurechnen, bietet Amazon eine Excel-Vorlage.\footcite{TCO_Comparison_Amazon_WorkSpaces_and_Traditional_Virtual_Desktop_Infrastructure_VDI_2014-11-15}

\subsection{Sicherheit und Geschwindigkeit}
Sicherheitsaspekte werden gut und weitgehend abgedeckt.
Es können Richtlinien aus dem \Gls{adLabel} übernommen werden.
Auf dem Client selbst werden keine Daten gespeichert.
Amazone bietet eine \Gls{glos:failureSafetyLabel} von 99,999999999 \% an. (Dies entspricht einer maximaler Ausfallzeit von 0.31536 Millisekunden/Jahr.)
Ein automatisches Backup der Benutzerdaten erfolgt alle 12 Stunden.

Für die Bildschirmübertragung (betrifft nur die effektiven Pixeln) werden die Daten komprimiert und verschlüsselt.
Nebst den Bildschirmdaten verlassen keinerlei Daten die \Gls{awsLabel} Server Infrastruktur.
Durch die Komprimierung ist eine hohe Auflösung der Bildschirminhalte möglich. Selbst bei einer langsamer Internetverbindung genügt das Bild zum Arbeiten.

\subsection{weitere Funktionen}
\begin{itemize}
	\item Das Interface ist für die mobile Nutzung optimiert.
	\item Bei Bereitstellung einer Workspace kann automatisch ein Mail, mit den nötigen Schritten, an den Enduser generiert werden.
	\item Für die Authentifizierung kann eine Multi-Factor Methode verwendet werden.
	\item Amazon selbst stellt den \textit{Amazon Zocalo Sync} zur Verfügung (entspricht der ähnlichen Funktion wie der Dropbox-Service\footcite{Dropbox_2014-11-15}).
	\item Die Verwendungen von lokalen Druckern ist möglich.
%	\item Der Client ist momentan nur in englischer Sprache verfügbar.
%	\item Zahlung erfolgt pro Monat und kann monatlich angepasst werden. Man muss grundsätzlich nicht für etwas zahlen, dass gar nicht verwendet wird.
	\item Für temporäre Arbeitnehmer kann sehr einfach eine Maschine für die Dauer der Anstellung gemietet werden, anschliessend kann der Zugriff wieder einfach und sicher entfernt werden.
\end{itemize}

\newpage
\section{VMware VDM}
% sl
% Quellen:
% - http://www.vmware.com/pdf/vdm20_intro.pdf
% - https://www.vmware.com/files/pdf/analysts/Forrester_Report_Total_Economic_Impact_of_VMware_Virtual_Desktop_Infrastructure_in_Financial_Services.pdf
VMware ist eine amerikanische Firma mit Sitz im Palo Alto, Californien. Gegründet wurde die Firma 1998 und ist seit 2007 an der Börse.
Als Hauptprodukt bietet VMware Virtualisierungssoftware für Server-Infrastrukturen an. Ein Teil davon, welchem sich dieses Kapitel widmet, ist die \Gls{vdiLabel}-Lösung \Gls{vdmLabel}, welche auf dem Kern von VMware, der \textit{VMware Infrastructure 3} aufbaut.

\subsection{Aufwand}
Der Initialaufwand für die Verwendung von \Gls{vdmLabel} ist relativ hoch, da die ganze Server-Infrastruktur mit VMware Produkten aufgesetzt werden muss.
Im Gegenzug gibt es einige Vorteile daraus:
\begin{itemize}
	\item Kontrolle und Management der Desktops aus einer Applikation heraus.
	\item Die Benutzer merken nichts von der Umstellung.
	\item Nahtlose Integration in die VMware Infrastructure Umgebung.
	\item Geringere Kosten als eine \gls{glos:onPremiseLabel} Lösung.
\end{itemize}

\clearpage

Die folgende Illustration zeigt das Setup, welches für den Betrieb von \Gls{vdmLabel} benötigt wird.
\begin{figure}[H]
	\includegraphics[width=\textwidth]{images/vmware-vdm-setup}
	\caption[VDM Setup]{VDM Setup \protect\footcite{Introduction_to_Virtual_Desktop_Manager}}
	\label{fig:vdmSetup}
\end{figure}

%cite: http://www.vmware.com/pdf/vdm20_intro.pdf

Zu Beginn werden die verschiedenen Terminals aufgelistet, mit denen man sich auf den persönlichen Desktop verbinden kann.

Unter der Netzwerk Wolke ist die Server Infrastruktur aufgezeigt.
Näher interessant für diese Arbeit sind die \textit{ESX Server hosts} in der unteren Mitte. Dabei handelt es sich um virtualisierte Server, welche jeder Zeit herauf- oder runterskaliert werden können.

Auf jedem ESX Hosts werden je mehrere virtuelle Desktops ausgeführt.
Auf den Desktops kann grundsätzlich jedes Betriebsystem ausgeführt werden, für welches VMware ein \Gls{vdmLabel}-Agent zur Verfügung stellt.

%    Windows XP Professional with SP2 (English, Japanese, German)
%    Windows XP Professional with SP3 (English only)
%    Windows Vista Business Edition (English, Japanese, German)
%    Windows Business Ultimate Edition (English, Japanese, German)


\subsection{Kosten}
\label{sec:vmwareCosts}
Die Berechnung der Kosten gestaltet sich schwierig, da es diverse Einflussfaktoren gibt.
VMware veröffentlichte in 2008 ein 26-seitiges Dokument mit dem Titel "`Total Economic Impact Of VMware Virtual Desktop Infrastructure – Financial Services Industry"'\footcite{Forrester_Report}. Darin ist ein Vierjahresplan aufgestellt mit einer möglichen Berechnung der Kosten.

Im Dokument werden zuerst Kennzahlen, wie die Löhne und die Anzahl Server spezifiziert.

Als zweites werden die initialen Kosten analysiert. Darin enthalten sind Lizenzkosten, Anzahl Arbeitsstunden, sowie die Hardware die benötigt wird.

Zuletzt sind die Laufzeitkosten aufgeführt.

Total sind die Kosten mit \$3'648'355 beziffert.

\begin{table}[H]
	\centering
	\small\renewcommand{\arraystretch}{1.4}
	\rowcolors{1}{tablerowcolor}{tablebodycolor}
	%
	\captionabove[VMware VDI Preise]{VMware VDI Preise}
	%
	\begin{tabularx}{\textwidth}{X | r | r | r | r | r | r}
		\hline
		\rowcolor{tableheadcolor}
		\textbf{Position} & \textbf{Initial} & \textbf{Jahr 1} & \textbf{Jahr 2} & \textbf{Jahr 3} & \textbf{Jahr 4} & \textbf{Total} \\
		\hline
		Initiale Implementierungskosten & \$37'438 & \$27'038 &  &  &  & \textbf{\$64'476} \\
		Software Lizenzen und \linebreak
		Maintanance & \$9'000 & \$52'425 & \$57'188 & \$55'2425 & \$65'550 & \textbf{\$239'588} \\
		Hardwarekosten & \$365'000 & \$1'095'000 & \$680'00 & \$315'000 & \$315'000 & \textbf{\$2'770'000} \\
		Laufende Kosten für \linebreak
		Implementation und Support &  & \$137'271 & \$141'389 & \$145'631 & \$150'000 & \textbf{\$574'292} \\
		\hline
		\rowcolor{tableheadcolor}
		Total & \$411'438 & \$1'311'735 & \$878'577 & \$516'056 & \$530'550 & \textbf{\$3'648'355} \\
	\end{tabularx}
\end{table}
%Horizontale Linien sind zu lang... WIESOOO???!!!???

Mit \textasciitilde76\% machen die Hardwarekosten den grössten Anteil an den totalen Kosten aus.
Das Szenario beinhaltet zu Beginn 60 \Glspl{vdiLabel} und wächst über die vier Jahre auf 1'000 \Glspl{vdiLabel} an. Dies entspricht einer grossen Firma, für welche dreieinhalb Millionen eine tragbare Investition darstellt. Für kleinere Firmen muss entsprechend runterskalliert werden.

\subsubsection{Einsparungen}
Das Paper beinhaltet auch die erwarteten Einsparungen.

\begin{table}[H]
	\centering
	\small\renewcommand{\arraystretch}{1.4}
	\rowcolors{1}{tablerowcolor}{tablebodycolor}
	%
	\captionabove[VMware VDI Einsparungen]{VMware VDI Einsparungen}
	%
	\begin{tabularx}{\textwidth}{X | r | r | r | r | r}
		\hline
		\rowcolor{tableheadcolor}
		\textbf{Position} & \textbf{Jahr 1} & \textbf{Jahr 2} & \textbf{Jahr 3} & \textbf{Jahr 4} & \textbf{Total} \\
		\hline
		Outsourcing Einsparungen & \$908,160 & \$3,027,200 & \$4,540,800 & \$4,540,800 & \textbf{\$13,016,960} \\
		Help desk Einsparungen &  &  & \$239,037 & \$253,600 & \textbf{\$492,637} \\
		US contractor Einsparungen &  & \$378,400 & \$756,800 & \$1,135,200 & \textbf{\$2,270,400} \\
		Vermiedene IT Einstellungen &  &  &  & \$100,000 & \textbf{\$100,000} \\
		Vermiedene PC Anschaffungen &  & \$28,750 & \$28,700 & \$57,500 & \textbf{\$115,000} \\
		Bürofläche Einsparungen &  &  &  & \$24,000 & \textbf{\$24,000} \\
		\hline
		\rowcolor{tableheadcolor}
		Total & \$908,160 & \$3,434,350 & \$5,565,387 & \$6,111,100 & \textbf{\$16,018,997} \\
	\end{tabularx}
\end{table}

Die mit Abstand grösste Einsparung liegt mit \textasciitilde82\% beim Outsourcing.

\subsubsection{Fazit}
Die genannten Zahlen sind mit einer Priese Salz zu geniessen. Als Erstes wurden sie für den US Amerikanischen Markt erhoben. Zweitens sind sie sicherlich sehr optimistisch berechnet, da sie aus dem Haus von VMware selbst stammen.

Dennoch stellen die Zahlen gut dar, dass es eine grosse Investition bedeutet für die es einen guten Grund geben muss.

Aus der Einsparungs-Tabelle kann man schlussfolgern, dass sich der Umstieg auf \Gls{vdiLabel} erst richtig lohnt, wenn man auch plant Outsourcing zu betreiben. Denn \textasciitilde82\% der Einsparungen beruhen auf diesem Punkt. Natürlich sind damit noch weitere Kosten verbunden die hier nicht aufgeführt sind.

\subsection{Sicherheit}
VMware integriert verschiedene Sicherheitsmassnahmen\footcite{Introduction_to_Virtual_Desktop_Manager} in \Gls{vdmLabel}.

\begin{itemize}
\item Zwei-Faktor-Authentifizierung
\item \Gls{httpsLabel} Zugang
\item \Gls{dmzLabel}
\item Load balancing
\item \Gls{ldapLabel}
\end{itemize}

\subsubsection{Zwei-Faktor-Authentifizierung}
\Gls{vdmLabel} unterstützt eine Zwei-Faktor-Authentifizierung mit RSA Secure ID. Beim Login muss der User Benutzername, Passwort und ein RSA-Token eingeben. Dieser Vorgang ist bekannt von E-Banking Plattformen.

Die Sicherheit besteht darin, dass wenn das Passwort in die falschen Hände gerät, sich der Angreifer nicht einloggen kann, da ihm das stetig ändernde RSA Token fehlt.

\subsubsection{HTTPS Zugang}
Der Zugang von den Clients auf die \Gls{vdmLabel} Infrastruktur sowie die Kommunikation zwischen den \Gls{vdmLabel} Servern selber kann gänzlich auf \Gls{httpsLabel} umgestellt werden.

\begin{minipage}{\textwidth}
\subsubsection{DMZ \& Load balancing}
Um den Zugriff auf die \Gls{vdiLabel} Umgebung aus dem Internet zu gewährleisten, kann eine \Gls{dmzLabel} eingerichtet werden\footcite{Introduction_to_Virtual_Desktop_Manager}.

\begin{figure}[H]
	\includegraphics[width=\textwidth]{images/vmware-vdm-dmz2}
	\caption{VDM Setup mit einer DMZ}
	\label{fig:vdmSetupDmz}
\end{figure}
\end{minipage}

Die \Gls{dmzLabel} ist ein Bereich, der vom Internet her zugänglich ist und dessen Server nicht gänzlich als vertrauenswürdig eingestuft werden können.

\Gls{vdmLabel} ermöglicht es verschiedene Security Server einzurichten. Dabei können welche in der \Gls{dmzLabel} stehen und andere ausserhalb.

\subsubsection{Load balancing}
Um die Ausfallwahrscheinlichkeit zu minimieren, kann vor die Security Servers ein Load balancing eingeschaltet werden. Siehe Abbildung \ref{fig:vdmSetupDmz}. Dieses verteilt unter Last die Anfragen an die Server. Fällt einer aus, merken die User, ausser eventueller Leistungseinbrüchen, nichts davon.

\subsubsection{\Gls{ldapLabel}}
Die Authentifizierung kann in das \Gls{ldapLabel} der Firma integriert werden. Die Benutzung wird dadurch vereinfacht, da die Benutzer sich mit den gewohnten Login-Daten anmelden können.

Auch die Sicherheit wird erhöht, da bei Neueintritten, internen Wechseln und Austritten die Rechte der Benutzer berücksichtigt werden.

\subsubsection{Fazit}
VMware integriert viele Sicherheitsaspekte in ihre \Gls{vdiLabel} Lösung.

Korrekt Umgesetzt kann damit eine sichere Infrastruktur aufgebaut werden, welche robust gegenüber Ausfällen und Häckerattacken ist. Ein zusätzlicher Vorteil ist, dass die Daten in der Firma gehosted sind und die Sicherheitszone niemals verlassen.

Alles steht und fällt jedoch mit der Integration in die eigene Infrastruktur. Die Mechanismen müssen korrekt und durchgehen eingehalten werden damit sie effektiv sind.

Der Zugriff über HTTP sollte durchgängig deaktiviert werden, wenn er nicht ausdrücklich benötigt wird. Ebenfalls sollte eine sichere Firewall nach aussen hin eingerichtet sein.

\subsection{Geschwindigkeit}
Gemessen werden kann die Geschwindigkeit an der Latenz zwischen der Eingabe am \Gls{vdiLabel} Client bis zum Zeitpunkt, wenn die Antwort auf dem Bildschirm erscheint. Als Richtwert für eine genügende Geschwindigkeit wird für eine normale Benutzung eine Sekunde vorausgesetzt und für ein- und ausgabelastige Prozesse sechs Sekunden.

VMware führte im März 2014 ein Lasttest durch, auf dessen Resultate sich dieser Abschnitt bezieht\footcite{VDI_Performance}.

\subsubsection{Testsetup}
Das Testsetup beinhaltet zwei Rechner. Der erste Rechner ist ein HP ProLiant BL460 mit einem 8-core Xeon E5540 (2.53 GHz) Prozessor und 96GB Ram. Als zweiter Rechner wurde ein Dell PowerEdge R720xd mit einem 16-core Intel Xeon E5-2690 (2.9 GHz) mit 256GB Ram verwendet.
Beide verfügen über eine \Gls{ssdLabel} mit 200GB und sechs 300GB 15k RPM SAS Festplatten.

Auf den Rechnern können beliebig viele Nodes hochgefahren werden, die sich die Last, die durch die Clients auftritt teilen.

Die Benutzer, die die Clients bedienen, sind automatisiert und produzieren eine hohe Last, welche \Gls{cpuLabel} und Ein-/Ausgabe lastig ist.

Die Tests werden jeweils mit 3, 5, 8 und 16 Nodes durchgeführt. Dadurch soll ersichtlich werden, ob das System mit zunehmender Nodes- und Benutzeranzahl gut skaliert.

\subsubsection{Resultate}
Der erste Test skalierte die Anzahl an Nodes von den definierten 3 bis zu 16 hoch und analysierte, wie viele Benutzer gleichzeitig unter den definierten Bedienungen arbeiten konnten.

\begin{figure}[H]
	\includegraphics[width=\textwidth]{images/vmware-vdm-scale}
	\caption{Skalierung von 3 Nodes bis zu 16 Nodes}
	\label{fig:vdmPerformanceScale}
\end{figure}

Die Grafik zeigt, dass 305 Clients flüssig auf 3 Nodes arbeiten konnten. Bei 16 Nodes waren es 1615 Benutzer. Zusätzlich ist ein linearer Anstieg ersichtlich. Das heisst, das System skaliert mit den Anzahl an Nodes.

Die zweite Grafik zeigt die Latenz für eine normale Benutzung mit der Anzahl an Benutzer aus der Abbildung \ref{fig:vdmPerformanceScale}.
\begin{figure}[H]
	\includegraphics[width=\textwidth]{images/vmware-vdm-performance-normal}
	\caption{Antwortzeiten für normale Benutzung}
	\label{fig:vdmPerformanceNormal}
\end{figure}

Auf der letzten Grafik wird die Antwortzeit für Ein-/Ausgabe lastige Bedienung illustriert.
\begin{figure}[H]
	\includegraphics[width=\textwidth]{images/vmware-vdm-performance-io}
	\caption{Antwortzeiten für Ein-/Ausgabe intensive Operationen}
	\label{fig:vdmPerformanceIo}
\end{figure}

Aus den Abbildungen \ref{fig:vdmPerformanceNormal} und \ref{fig:vdmPerformanceIo} bestätigt sich, dass sich das System gleich verhaltet, wenn die Anzahl an Nodes und damit die Anzahl an Clients erhöht wird.

\subsubsection{Fazit}
Mit genug starker Hardware kann jede gewünschte Antwortzeit erreicht werden. Bei VMwares \Gls{vdmLabel} Lösung sind das durchschnittlich 101 Benutzer pro Node, mit einer Toleranz von einer, beziehungsweise sechs, Sekunden.

Das spannendste ist jedoch die lineare Entwicklung mit zunehmender Nodeanzahl. Damit ist das System für kleinere Firmen gleich gut geeignet wie für grosse.
