% !TeX encoding=utf8
% !TeX spellcheck = de_CH_frami

%%% --- Acronym definitions
\IfDefined{newacronym}{%
% examples

% our used acronyms
\newacronym{apiLabel}{API}{Application Programming Interface}
\newacronym{crtpLabel}{CRTP}{Crazy RealTime Protocol}
\newacronym{ebsLabel}{EBS}{Einscheibe und Bewertungssystem der ZHAW}
\newacronym{vmLabel}{VM}{Virtual Machine}
\newacronym{vrLabel}{VR}{Virtual Reality}
\newacronym{zhawLabel}{ZHAW}{Zürcher Hochschule für Angewandte Wissenschaften}

}

%%% --- Symbol list entries

%\newglossaryentry{symb:Pi}{%
%  name=$\pi$,%
%  description={mathematical constant},%
%  sort=symbolpi, type=symbolslist%
%}


%%% --- Glossary entries
\newglossaryentry{glos:droneLabel}{
	name=Drohne,
	plural={Drohnen},
	description={Als Drohne wird ein unbemanntes Luftfahrzeug bezeichnet. Die Steuerung kann entweder manuell oder autonom erfolgen.}
}
\newglossaryentry{glos:dhlLabel}{
	name=DHL,
	description={Paket- und Brief-Express Dienst der Deutschen Post AG. Gegründet von Adrian \textbf{D}alsey, Larry \textbf{H}illblom und Robert \textbf{L}ynn.}
}
\newglossaryentry{glos:vtolLabel}{
	name=VTOL,
	description={Mit "`\textit{\textbf{v}ertical \textbf{t}ake-\textbf{o}ff and \textbf{l}anding}"' (VTOL) werden Luftfahrzeuge bezeichnet, die senkrecht Starten und Landen können.\footcite{Senkrechtstart_und_-landung__Wikipedia_2015-03-22}}
}
\newglossaryentry{glos:frameLabel}{
	name=Frame,
	plural={Frames},
	description={Als Frame, wird in vorliegendem Dokument, werden die Daten der Auswertung und Erkennung von Gesten innerhalb eines Zustandes (Bildes) bezeichnet.}
}









% use it with \gls{glos:DVD}
% use plural with \glspl{thinClientLabel}
