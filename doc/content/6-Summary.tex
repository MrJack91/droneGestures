\chapter{Schlussfolgerung}
Die Drohne kann mit Gestensteuerung zu geflogen und gesteuert werden.
Auf Youtube ist ein Video eines Fluges vorhanden.
\footcite{Crazyflie_20_with_Leap_Motion_Gesture_controlled_YouTube_2015-09-04}

Es wurden alle Punkte der Aufgabenstellung (im \secref{sec:workdef}) umgesetzt.
Die ursprünglich geplante Sensoren-Charakterisierung der Drohne wurde weggelassen, da die Sensoren für die jetzige Steuerung nicht gebraucht werden.\footnote{Dies wurde von der Betreuungsperson genehmigt.}
Ebenfalls wurden mögliche Erweiterungen zweiter Priorität (im \secref{sec:extensions} aufgeführt) weggelassen.

\section{Fazit}
Das Ziel, eine Drohne mit Gesten zu steuern, wurde erreicht.
Dabei wurde jedoch die Einfachheit einer Gestensteuerung überschätzt.
Obwohl von Beginn an eine einfache und intuitive Steuerung geplant war, zeigt sich die Steueraufgabe einer Drohne für schwieriger als erwartet.
Die Bedienbarkeit ist zwar intuitiv, aber die Drohne genau zu steuern ist auch mit der Gestensteuerung eine grosse Herausforderung.

Der Initialisierungsprozess, bei dem die Hand als Steuerung festgelegt wird, hat sich, mit den dokumentierten Änderungen im \secref{sec:poc:conceptChanges}, während mehreren Testflügen als sicher und brauchbar erwiesen.

Die Applikation ist \textit{Open Source} und stellt eine gute Basis für mögliche Weiterentwicklungen dar.


\section{Persönliches Schlusswort}
Für mich persönlich war diese Projektarbeit sehr spannend.
Dadurch, dass ich die Aufgabe grösstenteils selbst definieren durfte, konnte ich viele meiner Vorstellungen umsetzen.

Dabei gab es einige neue Herausforderungen:
Bis jetzt hatte ich praktisch keine Erfahrungen mit Hardware-Steuerungen, noch habe ich zuvor einen Gestensensor verwendet oder eine Drohne angesteuert.
Auch die Programmiersprache Python stellte für mich einen neuen Lerninhalt dar.

Mir hat diese Arbeit sehr zugesagt.
Durch die neuen Lerninhalte konnte ich einen Einblick in eine mir bisher verborgene Techniker-Ecke gewinnen.
