\IfPackageLoaded{caption}{%
% Style of captions
\DeclareCaptionStyle{captionStyleTemplateDefault}
[ % single line captions
  justification = centering
]
{ % multiline captions
% -- Formatting
  format    = plain,  % plain, hang
  indention  = 0em,   % indention of text 
  labelformat = default,% default, empty, simple, brace, parens
  labelsep   = colon,  % none, colon, period, space, quad, newline, endash
  textformat  = simple, % simple, period
% -- Justification
  justification = justified, %RaggedRight, justified, centering
  singlelinecheck = true, % false (true=ignore justification setting in 
%single line)
% -- Fonts
  labelfont  = {small,bf},
  textfont   = {small,sf},
% valid values:
% scriptsize, footnotesize, small, normalsize, large, Large
% normalfont, ip, it, sl, sc, md, bf, rm, sf, tt
% singlespacing, onehalfspacing, doublespacing
% normalcolor, color=<...>
%
% -- Margins and further paragraph options
  margin = 10pt, %.1\textwidth,
  % width=.8\linewidth,
% -- Skips
  skip    = 10pt, % vertical space between the caption and the figure
  position = auto, % top, auto, bottom
% -- Lists
  % list=no, % suppress any entry to list of figure 
  listformat = subsimple, % empty, simple, parens, subsimple, subparens
% -- Names & Numbering
  % figurename = Abb. %
  % tablename  = Tab. %
  % listfigurename=
  % listtablename=
  % figurewithin=chapter
  % tablewithin=chapter
%-- hyperref related options
  hypcap=true, % (true, false) 
  % true=all hyperlink anchors are placed at the 
  % beginning of the (floating) environment
  %
  hypcapspace=0.5\baselineskip
}

% apply caption style
\captionsetup{
  style = captionStyleTemplateDefault % base
}

% Predefinded skip setup for different floats
\captionsetup[table]{position=top}
\captionsetup[figure]{position=bottom}

\newcommand\FigureAbbrevition{Fig.}
\IfDefined{iflanguage}{%
  \iflanguage{ngerman}{%
    \renewcommand\FigureAbbrevition{Abb.}
  }{}
}

\DeclareCaptionStyle{captionStyleTemplateShortDefault}{%
  style=captionStyleTemplateDefault,
  name=\FigureAbbrevition,
  indention=0pt,
  justification=RaggedRight
}

% Short Names 
\IfDefined{wrapfigure}{%
  \captionsetup[wrapfigure]{style=captionStyleTemplateShortDefault}}
\IfDefined{wrapfloat}{%
  \captionsetup[wrapfloat]{style=captionStyleTemplateShortDefault}}
\IfDefined{floatingfigure}{%
  \captionsetup[floatingfigure]{style=captionStyleTemplateShortDefault}}
\IfDefined{margincap}{%
  \IfDefined{preto}{\preto\margincap{
  \captionsetup{style=captionStyleTemplateShortDefault}}}}
  % see http://tex.stackexchange.com/questions/37721/captionsetup-for-margin-caption
  % for an explanation of the extra code.
  %
} % end \IfPackageLoaded{caption}

% options for subcaptions
\IfPackageLoaded{subcaption}{
  \captionsetup[sub]{ %
    style = captionStyleTemplateDefault, % base
    labelfont  = {footnotesize,bf},
    textfont   = {footnotesize,sf},
    justification = RaggedRight, %RaggedRight, justified, centering
    skip=6pt,
    margin=5pt,
    labelformat = simple,% default, empty, simple, brace, parens
    labelsep    = space,
    list=false,
    hypcap=false
  }
  % make subcaptions be referenced as 5.3(b)
  \renewcommand\thesubfigure{(\alph{subfigure})} 
}

% style options for subfig
\IfPackageLoaded{caption}{%
 \IfPackageLoaded{subfig}{%
  \captionsetup[subfloat]{%
   style = captionStyleTemplateDefault, % base
   skip=6pt,
   margin=5pt,
   labelformat = parens,% default, empty, simple, brace
   labelsep    = space,
   list=false,
   hypcap=false
  }
 } % end \IfPackageLoaded{subfig}
} % end \IfPackageLoaded{caption}
